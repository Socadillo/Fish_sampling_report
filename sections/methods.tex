

\section{Materials and Methods}\label{sec:material_methods}          %% The first section

\subsection{Study area}\label{sec:study_area}                %% The Study areas




The benthic macro-invertebrate samples were taken at two sites near lake Lunz (Mostviertel, lower Ausrtria). For reference conditions the Ois River in Lunz am See was sampled~(\cref{fig:site_reference}). This small, natural, alpine river represents the headwaters of the Ybbs River and subsequently flows into the Danube river. The second sampling site was the so called Maiergraben, a small, concreted, channelized creek flowing through a forested area into lake Lunz~(\cref{fig:site_impactedA}).



\begin{figure}[!htb]                                                        %% Impacted site photo
\centering                                                                  %% Center the figures
\subcaptionbox{Heavy regulation present at impacted site.\label{fig:site_impactedA}}{   %% First subcaption
  \includegraphics[width=0.48\columnwidth]{images/site_impacted1}}    %% Set width and select first image COLOR
  \hfill                                                                                    %% Fill Blank space
\subcaptionbox{Sample collection from impacted area.\label{fig:site_impactedB}}{        %% Second subcaption
  \includegraphics[width=0.48\columnwidth]{images/site_impacted2}}        %% Set width and second image COLOR
  \hspace*{\fill}                                                                           %% Fill blank space
\caption{Impacted site located on Maiergraben.}\label{fig:site_impacted}          %% Capition for both figures
\end{figure}




\subsection{Methods}\label{sec:methods}                         %% The methods section

\subsubsection{Multi-Habitat Sampling (MHS)}\label{sec:mhs}         %% Describe MHS


For this approach the examined river reach was partitioned into 20 subunits. These sampling units were representing different habitat types (choriotopes) and their proportional areal coverage within the reach. Sampling units were characterized by both mineral~(\cref{tab:choriotope_mineral}) and biotic~(\cref{tab:choriotope_biotic}) habitats.


\begin{table}[!htb]                                 %% Mineral choriotope table
  \small                                                       %%makes the table font small
  \centering
  \caption{Mineral choriotopes.}
    \begin{tabular}{ l l p{7cm} }
  \toprule
    Nomenclature  &
    \multicolumn{1}{c}{Grain size}  &
    \multicolumn{1}{c}{Description of choriotope} \\
  \hline
  \hline
    megalithal      & \textgreater 40 cm               & upper sides of boulders, large cobbles and blocks, bedrock\\
    macrolithal     & \textgreater 20 cm to 40 cm      & coarse blocks, head-sized cobbles, variable percentages of cobbles, gravel and sand\\
    mesolithal      & \textgreater 6.3 cm to 20 cm     & fist to hand-sized cobbles and pebbles with a variable percentage of gravel and sand\\
    microlithal     & \textgreater 2 cm to 6.3 cm      & pebbles, coarse gravel with percentages of medium to fine gravel\\
    akal            & \textgreater 0.2 cm to 2 cm      & fine to medium-sized gravel\\
    psammal         & 0.063 mm to 2 mm      & sand\\
    pelal           & \textless 0.063 mm            & mud and sludge\\
    argillal        &                       & silt, loam and clay\\
  \bottomrule
    \end{tabular}
  \label{tab:choriotope_mineral}%
\end{table}%


First the share of mineral habitat classes was defined, secondly the biotic habitats within the mineral habitats were classified. This was done by a visual estimation of the area. For every 5\% of a certain choriotope (combination of mineral and biotic habitat) one sampling unit was chosen. The shares and subunits of each choriotope were then put in a field-protocol~(see~\hyperref[appendixA]{Appendix A}).

Sampling units also had to be distributed accordingly between the mesohabitats (river bottom/bank, lentic/lotic, riffles/pools). As the sampling area should not be disturbed beforehand, the sampling started downstream and then proceeded upstream.


%\paragraph{Habitat Composition}\label{sec:choriotopes}         %% Describe chiriotopes



\begin{table}[!htb]                                 %% Biotic choriotope table
  \centering
  \small                                                       %%makes the table font small
  \caption{Biotic choriotopes.}
    \begin{tabular}{ l l }
  \toprule
    Nomenclature  &
    \multicolumn{1}{c}{Description of choriotope} \\
  \hline
  \hline
    algal periphyton    & areal stone cover\\
    filamentous algae   & tufts or floating mats\\
    mosses              & -\\
    macrophytes         & submerged plants\\
    living wood         & roots, branches (with leaves),\newline tree trunks\\
    deadwood            & branches and tree trunks\\
    CPOM                & course particulate matter\\
    FPOM                & fine particulate organic matter\\
    sapropel            & decaying sludge\\
    bacteria \& fungi   & lawns and tufts\\
  \bottomrule
    \end{tabular}
  \label{tab:choriotope_biotic}%
\end{table}%










\subsubsection{Sampling Method}\label{sec:sampling_method}          %% Describe impacted site

\begin{table}[!htb]                                         %% Ois Choriotopw Table
  \small                                                       %%makes the table font small
%  \centering
  \caption{Choriotope description of the sampling units at the Ois River.}
  \resizebox{\textwidth}{!}{                                                          %% Resizes the table to the text
  \begin{tabular}{ r b{.12\textwidth} b{.15\textwidth} c r r r }
    \toprule
    \multicolumn{1}{R{.08\textwidth}}{Sample} &
    \multicolumn{1}{B{.13\textwidth}}{Mineral Habitat} &
    \multicolumn{1}{B{.17\textwidth}}{Biotic \par\noindent Habitat} &
    \multicolumn{1}{B{.15\textwidth}}{Velocity \par\noindent [m/s]} &
    \multicolumn{1}{B{.08\textwidth}}{Velocity Class} &
    \multicolumn{1}{B{.08\textwidth}}{Water Depth \par\noindent [cm]} &
    \multicolumn{1}{B{.1\textwidth}}{Distance \par\noindent to Shore\par\noindent [m]} \\
    \hline
    \hline
    1\_2 & mesolithal & micro\_algae & 0.9   & high  & 10    & 2 \\
    3\_4 & macrolithal & micro\_algae\_moss & 1.3   & very\_high & 25    & 6 \\
    5\_6 & megalithal & moss  & 1.3   & very\_high & 5     & 8 \\
    7\_8 & mesolithal & micro\_algae & 0.5   & medium & 20    & 4 \\
    9\_10 & megalithal & micro\_algae & 0.5   & medium & 10    & 8 \\
    11\_12 & mesolithal & periphyton & 0.2   & slow  & 30    & 5 \\
    13\_14 & akal & none  & 0.2   & slow  & 25    & 2 \\
    15\_16 & mesolithal & CPOM  & {  0}     & no\_flow & 20    & 0.5-1 \\
    17\_18 & macrolithal & micro\_algae & {  0}     & no\_flow & 10    & 0.5 \\
    19\_20 & mesolithal & CPOM  & 0.2   & slow  & 20    & 0.5 \\
    \bottomrule
    \end{tabular}}%
  \label{tab:choriotope_ois}%
\end{table}%



Sample collection at each sampling unit was performed with a stationary rectangular net (Mesh size: \SI{500}{\micro\meter} EU standard) and by disturbing a quadratic area upstream of the net (25x25cm). For the Ois River we chose 10 sampling locations where we took 2 samples each, for a total of 20 units~(\cref{tab:choriotope_ois}).




Since the Maiergraben had a very homogenous structure we chose only four sampling locations – three in the creek and one under a small bridge. We collected 5 samples from each location for a total of 20 sampling units.~(\cref{tab:choriotope_maiergraben,fig:site_impactedB})


\begin{table}[!htb]
  \small                                                       %%makes the table font small
  %\centering
  \caption{Choriotope description of the sampling units at the Maiergraben.}
  \resizebox{\textwidth}{!}{                                                          %% Resizes the table to the text
  \begin{tabular}{ r b{.12\textwidth} b{.18\textwidth} c r r r }
    \toprule
    \multicolumn{1}{R{.08\textwidth}}{Sample} &
    \multicolumn{1}{B{.13\textwidth}}{Mineral Habitat} &
    \multicolumn{1}{B{.17\textwidth}}{Biotic \par\noindent Habitat} &
    \multicolumn{1}{B{.15\textwidth}}{Velocity \par\noindent [m/s]} &
    \multicolumn{1}{B{.08\textwidth}}{Velocity Class} &
    \multicolumn{1}{B{.08\textwidth}}{Water Depth \par\noindent [cm]} &
    \multicolumn{1}{B{.1\textwidth}}{Distance \par\noindent to Shore\par\noindent [m]} \\
    \hline
    \hline
    1-5   & technomega & micro\_macro\_algae & 0.22-0.32 & medium & 4.8   & 0.5 \\
    6-10  & technomega & micro\_macro\_algae & 0.24-0.34 & medium & 2.8   & 0.5 \\
    11-15 & technomega & micro\_macro\_algae & 0.22-0.32 & medium & 4.5   & 0.5 \\
    16-20 & technomega & bare  & 0.30-0.40 & medium & 4.2   & 0.5 \\
    \bottomrule
    \end{tabular}}%
  \label{tab:choriotope_maiergraben}%
\end{table}%


Samples taken from organic habitats, always had to include the underlying mineral substrates. The different mineral habitats needed to be sampled accordingly.
\begin{list}{}
  \item {\textbf{Megalithal:} Boulders were sampled from all sides by sweeping the surface with a brush and flushing the animals into the net.}
  \item {\textbf{Macro- and Mesolithal:} Surface dwelling animals were flushed in the net by gently sweeping the cobbles or stones by hand. Next clingers and sessile animals were scratched off with a brush. Finally, the underlying substrate (within 15 to 20 cm depth) was churned by foot.}
  \item {\textbf{Microlithal and Akal:} Coarse gravel and sandy substrate was sampled by disturbing the sediment by kicking it downstream the net.}
\end{list}

After sampling each unit, the content of the net was transferred first into a tray and then into a closable bucket, which was filled with ethanol, in order to kill the animals.


\subsubsection{Sorting Techniques}\label{sec:sorting_technigues}      %% Describe impacted site

To sort the animals by size classes the buckets contents were later put through a sieve tower with different mesh sizes (\SI{10}{\milli\meter} down to \SI{500}{\micro\meter}). The different fractions were then placed in trays~(\cref{fig:sorting_tray}), so animals could be sorted, counted and determined to screening taxa level with the help of binoculars~(\cref{fig:sorting}). Only whole animals, so no single body parts, empty shells, exuvies or headless animals were taken into account. In trays with large amounts of animals (usually the smaller fractions) rare or single-occurring animals were taken out, and the rest was subsampled. This was done by transferring the animals in another tray that consisted of a grid, separating the tray into 16 or 8 cells. The organisms of usually two representing subsamples were determined and counted. The number of animals then had to be multiplied accordingly.
A taxa list was created with MS Excel and further analysis were made with the programs MS Excel and EcoProf (version 4.0.0).






\begin{figure}[!htb]                               %% Sample processing
\centering                                                                  %% Center the figures
\captionbox{Sorting tray used for taxa identification.\label{fig:sorting_tray}}{   %% First subcaption
  \includegraphics[height=7.6cm]{images/sorting_tray}}         %% Set width and select first image COLOR
  \hfill                                                                                    %% Fill Blank space
\captionbox{Identifying taxa within the samples using binoculars.\label{fig:sorting}}{     %% Second subcaption
  \includegraphics[height=7.6cm]{images/sorting}}        %% Set width and second image COLOR
  \hspace*{\fill}                                                                    %% Fill blank space
\end{figure}




\subsection{Equipment}\label{sec:equipment}                 %% The equipment section


\begin{table}[htbp]
  \centering
    \begin{tabular}{l b{3cm} l}
    ·         Nets & ·   &      Sieve tower \\
                &   &\\
    ·         Brushes & · &        Sorting trays~(\cref{fig:sorting_tray}) \\
                &   &\\
    ·         Trays & ·    &     Subsampling trays \\
                &   &\\
    ·         Buckets (closeable) & ·  &       Tweezers \\
                &   &\\
    ·         Ethanol & ·     &    Binoculars \\
    \end{tabular}%
  \label{tab:addlabel}%
\end{table}%


