

\section{Discussion}\label{sec:discussion}                 %% The first section

In the following part the results of the assessment will be discussed following the same order as in the results section.

\subsection{Microhabitats}\label{sec:microhabitats_discussion}       %% The first subsection of discussion


When comparing the two rivers Ois and Maiergraben, the differences between them are quite obvious. The Ois is in a very good ecological status according to both the organic and general degradation. The substrate size of the Ois ranged from sand to megalithal. Together with a broad range of flow velocities and water depths, benthic invertebrate taxa encounter different microhabitats to colonise. Meso- and macrolithal together occupy 70\% of the mineral microhabitats, which creates a big surface to colonise with a lot of interstitial space. Furthermore, different biotic cover is present at mesolithal habitats such as micro algae in medium (7\_8) to high flow velocities (1\_2) and periphyton in slow flow velocities (11\_12). At sampling sites with slow flow velocities (19\_20) or no flow (15\_16), aggregations of CPOM are present. The macrolithal contains micro algae/moss at very high flow velocities (3\_4) and micro algae where no flow is present (17\_18). The akal microhabitat has no biotic cover (13\_14), which can be explained by the instability of the substrate, whereas megalithal is covered by either moss (5\_6) or micro algae (9\_10). At the Ois, habitats for various benthic invertebrate taxa with different strategies are available. CPOM supplies grazers with food whereas micro and macro algae, moss and periphyton support grazers and other animals which cling to plant parts in order to cope with high currents. Low flow velocities provide habitats for swimming animals and others, which are not well adapted to high flow velocities. On the other hand do very high flow velocities support filter feeders, which can be observed at the sampling sites 1\_2 and 3\_4, where simulids occur in high numbers~(\cref{fig:taxa_ois}).

\begin{wrapfigure}{r}{0.55\textwidth}
\centering
\includegraphics[width=0.45\textwidth]{images/site_impacted3}
\caption{\label{fig:site_impacted3}Bridge covering section of Maiergraben.}
\end{wrapfigure}

According to the Screening Taxa analysis~(\cref{fig:ecoprof}), the Maiergraben is in a worse ecological status due to general degradation and therefore in need of action. The Maiergraben is framed by a corset of technomegalithal (\cref{fig:site_impactedA}), which lowers its habitat heterogeneity at both levels, the abiotic and biotic, when compared to the Ois. Due to steep stone walls at either side as well as at the river bed, the water depth is uniform and therefore formations of sections with lower flow velocity and deeper areas are highly restricted. Additionally, the lateral connectivity to the floodplain is no longer available. As can be seen in \cref{tab:choriotope_maiergraben}, the flow velocity is very monotonous and classified as medium and therefore CPOM and FPOM can easily be washed out. The good saprobity-status of the Maiergraben underlines this assumption.

The biotic habitat in the Maiergraben is rather monotonous too, whereby 15 out of 20 sampling points are overgrown by micro/macro algae. Clear water and low water depths allow penetration of sunlight through the water column, which makes plant growth possible. The remaining five sites without algal cover were located below a bridge~(\cref{fig:site_impacted3}) and therefore in the shadow and lay bare.


\subsection{Taxa Composition}\label{sec:taxa_composition_discussion}      %% The second section of discussion


The taxa composition of the Ois is diverse with 48 present taxa~(\cref{fig:ecoprof}), which can be explained by its habitat heterogeneity as discussed above. In the following section five different habitats 1\_2, 7\_8, 9\_10, 17\_18 and 19\_20, which differ most in their taxa composition, are discussed.

Sampling site 1\_2 has a high abundance of 1818 Ind./\SI{}{\square\meter} and is dominated by simulids, which are passive filter feeders~\parenciteA{Car2002} and their occurrence can be explained by high flow velocities and algal cover (\cref{tab:choriotope_ois}), which can be used to cling on in order to cope with high currents as seen in~\cref{fig:simulidae}. Dominating ephemeropterans are from the family Baetidae, which are assumed to hide in the micro algae cover or in the interstice of the mesolithal because due to their shape the adaption to high current is not very high. The second dominating ephemeropterans, \emph{Rhithrogena, sp.}, on the other hand are well adapted to high flow velocities and graze on algal cover or feed on detritus, which is trapped between plant parts~\parenciteA{Bauernfeind2002}. Furthermore, occurring plecopterans belong to Capniidae/Leuctridae, \emph{Amphinemura} sp. and \emph{Protonemura} sp., which are classified as grazer, shredders and detrivorous, as well as \emph{Isoperla} sp. and \emph{Dinocras} sp., which are predators.


\begin{figure}[!htb]                            %% Caddis fly cases
  \center
  \includegraphics[width=.75\linewidth]{images/simulidae}              %% Width, Image file COLOR
  \caption{Mass abundances of Simulidae (black flies) found at Ois river.}            %% Figure Caption
  \label{fig:simulidae}                                                        %% Figure label key
\end{figure}

Sampling site 7\_8 is not dominated by any taxa group and abundance is lower with 706 Ind./\SI{}{\square\meter}. The habitat is characterised by mesolithal and therefore a big surface to colonise, overgrown with micro algae and medium flow velocity (\cref{tab:choriotope_ois}). Ephemeropterans are represented mainly by Baetidae and \emph{Rhithrogena} sp., which are different according to their adaption to high flow velocities. \emph{Rhithrogena} sp. belongs to the family Heptagenidae, which have a characteristically flattened body shape, whereas Baetidae are rather slim. Both taxa are grazers and detritus feeders~\parencite{Bauernfeind2002}. Furthermore, coleopterans are represented by \emph{Elmis} sp. and \emph{Esolus/Oulimius/Riolus} sp., which are grazing, shredding or detrivorous and \emph{Hydraena} sp., which is grazing as an adult while clinging to stones or plant parts with its hooks and predatory as a larvae~\parenciteA{Jach2002}. Plecoptera are represented by the same taxa as at sampling site 1\_2 and show a similar pattern in abundances. Trichopterans are represented by the families Glossosomatidae, Limnephilidae and Sericostomatidae and also a small number of \emph{Brachycentrus montanus}, which is classified as passive filter feeder, as well as predatory and grazing~\parenciteA{Graf2002a}. Additionally, chironomids and other dipterans are present.

Sampling site 9\_10 is characterised by megalithal, which is overgrown with micro algae and medium flow velocity (\cref{tab:choriotope_ois}) and is clearly dominated by chironomids. Additionally, higher abundances of \emph{Protonemura} sp., Baetidae Gen. sp. and Elmidae Gen. sp. let assume that accumulations of detritus are available since all taxa are at least partly detritus feeding. At this sampling site together with site 3\_4 trichopterans are most diverse and represented by six taxa, namely Rhyacophilidae Gen. sp, Hydroptilidae Gen. sp., \emph{Hydroptila} sp., \emph{Hydropsyche} sp., Psychomyiidae Gen. sp. and \emph{Micrasema minimum}.

Sampling site 17\_18 has no flow and is characterised by macrolithal and micro algae (\cref{tab:choriotope_ois}). The dominating group are ephemeropterans, mainly due to \emph{Ephemerella} sp., which are grazing and detrivorous~\parenciteA{Bauernfeind2002} and seem to thrive with no flow and growth of micro algae because they account for almost 50\% of all individuals. This site is also the only one where oligochaets are present, represented by \emph{Nais} sp.

The last sampling site to be discussed is site 19\_20, which is characterised by mesolithal, slow flow velocity and CPOM cover (\cref{tab:choriotope_ois}). More than 95\% of occurring taxa belong to the EPT-Taxa; however, the total abundance is lowest with 236 Ind./\SI{}{\square\meter}. Dominating plecopterans are Capniidae/Leuctridae and \emph{Amphinemura} sp., which are classified at least partly as shredders~\parenciteA{Graf2002}. The most dominant ephemeropteran taxa are again \emph{Ephemerella} sp., indicating their preference for little current. \emph{Ecdyonurus} sp. and \emph{Baetis muticus} are occurring too, which are characterised as grazing and detrivorous~\parenciteA{Bauernfeind2002}. Since the water level is low algal cover is present and slow flow velocities together with availability of CPOM encourage formation of detritus. The present trichopterans are Limnephilidae Gen. sp., Sericostomadidae Gen. sp. and \emph{Potamophylax rotundipennis}. The latter one is mainly shredding organic material and to small parts also grazing and predatory.

However, a pattern of lower abundances at sampling sites 13\_14 to 19\_20 (they range between 236-424 Ind./\SI{}{\square\meter}), where flow velocities are either slow or missing, can be observed. A possible explanation can be a lower availability of food items, whereas habitats with higher flow velocities are supplied continuously. At sites where CPOM is available this assumption might not be true. The site 11\_12 is the only one with slow flow velocity and a higher abundance (924 Ind./\SI{}{\square\meter}), but also the only site which is covered with periphyton. The high abundance is due to domination of chironomids (406 Ind./\SI{}{\square\meter}), which seem to be well adapted.

At Maiergraben the taxa composition is rather similar at all sampling sites (\cref{fig:taxa_maiergraben}) and in total only 24 different taxa are present~(\cref{fig:ecoprof}). Interestingly, abundances differ greatly from 182 Ind./\SI{}{\square\meter} at site 16-20 to 2910 Ind./\SI{}{\square\meter} at site 6-10. The site 16-20 is also the only one where no biotic cover is present and therefore an important microhabitat and food source is missing. The dominating group are always chironomids followed by ephemeropterans except at sampling site 6-10, where simulids are the second most abundant group. At site 6-10 chironomids (2011 Ind./\SI{}{\square\meter}) and simulids (464 Ind./\SI{}{\square\meter}) together account for almost 2500 Ind./\SI{}{\square\meter}. A reason for high simulid abundances could be the very low water depth of 2.8 cm (\cref{tab:choriotope_maiergraben}), which makes filtering easy because food is always close. Ephemeropterans are represented by Baetidae Gen. sp. at all sites except of an additional specimen (1 Ind./\SI{}{\square\meter}) of \emph{Rhithrogena} sp. at site 6-10. Trichopterans are more diverse, but most dominant are the taxa Rhyacophilidae Gen. sp., Rhyacophila s. str. Sp. and Rhyacophila aquitanica/tristis, which belong to the predators~\parencite{Graf2002a} and seem to benefit from the high abundances of chironomids. Coleopterans at all sites belong to Esolus/Oulimnius/Riolus sp., which are classified as grazing, shredding and detrivorous~\parencite{Jach2002} and might feed as grazers at the sites 1-15 and as detritus feeders at site 16-20.




%\subsection{EPT-Taxa}\label{sec:ept_taxa_discussion}                      %% The third section of discussion
\subsubsection{EPT-Taxa}\label{sec:ept_taxa_discussion}                      %% The third section of discussion

The abundance of EPT-Taxa is shown for both the Ois and Maiergraben in \cref{fig:EPT}. In total 29 EPT-Taxa are present at the Ois and 12 at the Maiergraben. Additionally, many more individuals are present at the Ois, which is especially true for plecopterans, where individuals of Capniidae/Leuctridae, \emph{Amphinemura} sp. and \emph{Protonemura} sp. are often among the dominating taxa. The same is not true for the Maiergraben, where abundances of plecopterans are generally low. One possible explanation is the lack of interstices, which are often used by those taxa. Furthermore, those groups show highest abundances at sample sites 3\_4 and 5\_6 with very high flow velocity and the presence of moss. At the Maiergraben no predatory plecopterans are present, but many predatory trichopteran taxa are. Moreover, abundances of trichopterans are very similar at both rivers; however, the taxa are more diverse at the Ois, where 14 taxa are present compared to seven at Maiergraben. This fits to the assumption that in impacted rivers less taxa, but higher abundances occur compared to the many taxa and lower abundances found in intact stretches (see~\hyperref[appendixB]{Appendix B} ). The Ois shows higher number of taxa and higher abundances of ephemeropterans too. At the Maiergraben only two ephemeropteran taxa are present whereby \emph{Rhithrogena} sp. has an abundance of 1 Ind./\SI{}{\square\meter}. Furthermore, Baetidae Gen. sp. are given a BMWP-Score of only 4, which indicates that they are not very sensitive~\parenciteA{Martin2007}. Even though all flow velocities at Maiergraben are classified as medium only little ephemeropteran taxa, which are adapted to those flow velocities, are present. At the Ois, on the other hand, many different taxa are present with various adaptations to their environment.

The EPT-taxa distribution between the different choriotopes provides insight into their habitat preferences.
In general, mesolithal shows the highest diversity (\cref{fig:EPT_choriotope}). As shown in \cref{fig:choriotope_composition}, it is the most abundant mineral habitat at the Ois, covering 50\% in total. Furthermore, different biotic cover such as micro algae, periphyton and CPOM is present, as well as a broad range of flow velocities such as no flow, slow flow, medium flow and high flow (\cref{tab:choriotope_ois}). Hence, the mesolithal provides very different habitats, which meet a variety of needs together with a high surface and a lot of interstitial space. On the other hand, akal accounts for only 10\% of available habitat, has no biotic cover and only occurs at slow flow, which excludes permanent colonisation by many rheophilic species.


%\subsection{Sensitive Taxa}\label{sec:sensitive_taxa_discussion}          %% The fourth section of discussion
\subsubsection{Sensitive Taxa}\label{sec:sensitive_taxa_discussion}          %% The fourth section of discussion

The number of sensitive taxa differs between both rivers. In the Ois 27 taxa (56\%) are present, whereas in the Maiergraben only six taxa (25\%) are present, which is one reason for the low score at the general degradation (\cref{fig:ecoprof}). On average, 12.4 sensitive taxa are present at a site in the Ois and only 3.25 at the Maiergraben respectively. Furthermore, as shown in \cref{fig:sensitive_percentage,fig:sensitive_individuals}, the number of sensitive individuals is low in relation to the number of sensitive taxa. For example, even though 56\% of all taxa in the Ois are classified as sensitive, they only account for 37\% of all individuals. The same can be observed in the Maiergraben, but the difference is more drastic. While 25\% of all taxa are classified as sensitive, they account for only 4\% of all individuals (\cref{fig:sensitive_percentage}). This result underlines the hypothesis that non-sensitive taxa, and especially those which can cope with a variety of environmental conditions, occur in low taxa richness but high abundances, whereas in unimpacted stream sensitive taxa occur in sometimes high diversity and taxa richness but low abundances.


At the Ois, sensitive taxa belong to the groups Coleoptera, Diptera, Ephemeroptera, Plecoptera and Trichpotera. Among them \emph{Amphinemura} sp., \emph{Protonemura} sp., \emph{Rhithrogena} sp., \emph{Ephemerella} sp., \emph{Hydraena} sp., \emph{Elmis} sp. and \emph{Micrasema minimum} show high abundances. At the Maiergraben, on the other hand, sensitive taxa belong to the groups Cleoptera, Diptera, Ephemeroptera and Plecoptera and except for Elmidae Gen. sp. and \emph{Protonemura} sp. abundances are very low. Sensitive taxa benefit from higher habitat diversity and more colonisable surfaces at the Ois, whereas low habitat diversity and less colonisable surface is available at the Maiergraben. This is due to the technomegalithal substrate and total lack of interstitial space. Competition with taxa which are well adapted to conditions found at the Maiergraben could be another stressor for sensitive taxa.


