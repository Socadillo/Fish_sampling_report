

\section{Conclusion}\label{sec:conclusion}                                                   %% The first section

Many differences are obvious when comparing the impacted Maiergraben to the unimpacted Ois. The low degradation score of the Maiergraben (\cref{fig:ecoprof}) can be explained by its channelization and therefore loss of habitat heterogeneity. The technomegalithal riverbed of the Maiergraben is very monotonous and does not provide many structures and has a total lack of interstitial substrate. Additionally, the shallow water flows at a medium velocity, leading to a very similar habitat throughout the river stretch. This lack of habitat diversity results in less taxa and among them less EPT- and sensitive taxa, as well as lower total abundances. The Ois on the other hand provides a variety of different habitats, containing various mineral habitats, biotic cover and flow velocities. High availability of colonisable habitats and food supports a diverse benthic invertebrate fauna.


The low level of organic pollution means that the Maiergraben has little to no organic degradation, however the general degradation is severe enough that there is a need for action. Habitat availability must be increased along with providing more diversity among available habitats. Therefore appropriate measures such as river widening and restoring connectivity to the floodplain should be implemented. These measures would provide the Maiergraben with a more natural flow regime, which would greatly enhance the benthic habitat heterogeneity. Providing new habitat would increase the number of taxa occurring in the river and improve the general degradation score of the Maiergraben.