\section{Results}\label{sec:results}




\subsection{Screening Method}\label{sec:screening_method}       %% Screening Methond

A comparison of the river Ois and the Maiergraben was made using EcoProf~(\cref{fig:ecoprof}). Both rivers are located in the bioregion limestone prealps and the basic condition for general degradation is set to 1.5. More screening taxa in general and among them more sensitive taxa were found in the Ois than in the Maiergraben (see~\hyperref[appendixB]{Appendix B} for full taxon lists). Furthermore, the degradation-score is very high for the Ois, but the Maiergraben does not reach the expected value of 75. The saprobity-score is below 1 for the Ois and just above 1 for Maiergraben, which indicates that organic pollution is not a significant problem for either river. EcoProf showed that the Ois has very good status, while there is a need for action in the Maiergraben. This is due to general degradation as suggested by the low number of occurring sensitive taxa and the low degradation score.

\begin{figure}[!htb]                              %% EcoProf
  \center
  \includegraphics[width=.8\linewidth]{images/ecoprof}                %% Width, Image file COLOR
  \caption{EcoProf output for the Screening Method comparing the river Ois with Maiergraben.}        %% Figure Caption
  \label{fig:ecoprof}                                                       %% Figure label key
\end{figure}




\subsection{Microhabitats}\label{sec:microhabitats_results}       %% Microhabitats

The choriotope composition of the sites can be found in~\cref{fig:choriotope_composition} with more general habitat characteristics available in~\cref{tab:choriotope_ois,tab:choriotope_maiergraben}. From the first view it is obvious that in the Ois more diverse mineral habitats and biotic choriotopes are available. In the Ois the dominating mineral habitat is the mesolithal with a share of 50\% followed by macro- and megalithal with a share of 20\% each. Additionally, a small proportion with a share of 10\% of the river stretch is covered with sand. On the other hand, in the Maiergraben only technomegalithal is available for colonisation, making the river stretch very monotonous.


\begin{figure}[!htb]                               %% Choriotope composition
  \center
  \includegraphics[width=.9\linewidth]{images/choriotope_mineral}                 %% Width, Image file COLOR
  \includegraphics[width=.9\linewidth]{images/choriotope_biotic}                 %% Width, Image file COLOR
      \caption{Habitat Composition: The mineral choriotope composition of Ois (upper left) and Maiergraben (upper right) and the biotic choriotope composition of Ois (lower left) and Maiergraben (lower right).}      %% Figure Caption
  \label{fig:choriotope_composition}                                                        %% Figure label key
\end{figure}


The lower pie-charts in~\cref{fig:choriotope_composition} show the biotic choriotope composition of the rivers. In the Ois, the largest percentage of mineral habitats are covered by micro algae with a share of 40\% and CPOM with 20\%. The remaining river stretch is covered by equal parts of periphyton, moss or micro algae/moss. The Maiergraben shows less diverse biotic choriotopes and is dominated by micro/macro algae which cover 15 out of 20 sampling sites. The remaining five sampling sites lie bare.



\subsection{Taxa Composition}\label{sec:taxa_composition_results}       %% Taxa compsition

In the following section the taxa composition of the rivers Ois and Maiergraben will be compared in general and in more detailed between the individual sampling sites. This will be followed by a comparison of ephemeroptera, plecoptera and trichoptera taxa (EPT) and then Sensitive Taxa.


%\subsubsection{Taxa Composition at the rivers Ois and Maiergraben}\label{sec:taxa_composition_sites}  %% Taxa compsition

The taxa composition between the Ois and Maiergraben varied greatly~(\cref{fig:taxa}). The most dominant group in the river Ois are the dipterans with a share of 43\%; among them chironomids are slightly dominant with 23\% over simulids with 18\% and other dipterans with a share of 2\%. The second most abundant group are plecopterans with a share of 28\% followed by ephemeropterans with 14\%. Trichopterans have a share of 4\% which gives a total share of EPT-Taxa of 46\% at the Ois. The remaining part of the occurring taxa constitutes of coleopterans with a share of 8\% and other groups referred to as rest.

The Maiergraben, on the other hand, paints a quite converse picture. Similar to the Ois, dipterans are the dominating group in the Maiergraben with a much bigger share of 79\% in total. Among them, chironomids are dominating with a share of 68\% followed by simulids with 10\% and other dipterans with 1\%. Compared to the Ois the share of plecopterans is very low with only 1\%. Ephemeropterans have a share of 9\% and trichopterans have a slightly bigger share of 6\% when compared to the Ois. In general, EPT-Taxa account for only 16\% of the Maiergraben taxa composition, which is much lower than in the Ois. Additionally, 3\% of the occurring taxa is made up by coleopterans and the remaining 2\% by other groups.

\begin{figure}[!htb]                              %% Composition taxa
  \center
  \includegraphics[width=.7\linewidth]{images/taxa}                 %% Width, Image file COLOR
  \caption{Composition of taxa at sampling sites.}                        %% Figure Caption
  \label{fig:taxa}                                                        %% Figure label key
\end{figure}

Furthermore,~\cref{fig:taxa_composition} allows a more detailed look on the taxa composition of the two rivers. The first impression is that the taxa composition is very diverse between the different sampling sites in the Ois when compared to the Maiergraben, where the taxa composition looks very similar at the different sampling sites.

A closer look to the river Ois~(\cref{fig:taxa_ois}) shows that the sampling site {1\_2} is dominated by simulids, but ephemeropterans and plecopterans account for a large part of the groups. Taxa composition at sampling site 3\_4 looks similar, but the share of simulids is less and plecopterans increase. At sampling site {5\_6} simulids are replaced by chironomids and the share of plecopterans increases again. Sampling site {7\_8} has a substantial share of EPT-Taxa with a domination of ephemeropterans and many coleopterans occur as well. Sampling site {9\_10} shows a clear domination of chironomids followed by EPT-Taxa and a high number of coleopterans. At sampling site {11\_12} the most abundant groups are chironomids and plecopterans, which is similar to the sites 13\_14 and 15\_16. At the sampling site {17\_18} the dominant group are ephemeropterans followed by plecopterans and chironomids. Sampling site {19\_20} is dominated by plecopterans and ephemeropterans and has the highest share of EPT-Taxa among all sampling sites, as well as the lowest number of occurring groups with five different ones.


\begin{figure}[!htb]                                    %% Taxa composition of sites
\centering                                                                  %% Center the figures
\subcaptionbox{Composition of Ois sub-samples.\label{fig:taxa_ois}}{   %% First subcaption
  \includegraphics[width=0.48\columnwidth]{images/taxa_ois}}         %% Set width and select first image COLOR
  \hfill                                                                                    %% Fill Blank space
\subcaptionbox{Composition of Maiergraben sub-sambles.\label{fig:taxa_maiergraben}}{        %% Second subcaption
  \includegraphics[width=0.48\columnwidth]{images/taxa_maiergraben}}        %% Set width and second image COLOR
  \hspace*{\fill}                                                                           %% Fill blank space
\caption{Taxon composition of reference (a) and impacted (b) sampled sites.}\label{fig:taxa_composition}          %% Capition for both figures
\end{figure}

The taxa composition at the Maiergraben~(\cref{fig:taxa_maiergraben}) looks very similar between all sampling sites. Generally, chironomids are the dominating group and the share of EPT-Taxa is low (also compare with~\cref{fig:EPT}). Among the EPT-Taxa ephemeropterans are dominating followed by trichopterans and a very low number of plecopterans. At the sampling sites 11-15 and 16-20 the share of coleopterans is noteworthy, as well as the bigger share of simulids at sampling site 6-10 in comparison to the others.






\subsubsection{EPT-Taxa}\label{sec:ept_taxa_results}  %% EPT Taxa compsition
%\subsection{EPT-Taxa}\label{sec:ept_taxa_results}  %% EPT Taxa compsition

There was a much higher occurrence of EPT-taxa in the Ois when compared with the Maiergraben~(\cref{fig:EPT}), which is true for all of the groups. Very interesting is the huge difference between the numbers of plecopterans, whereby only 50 Ind./\SI{}{\square\meter} are present at the Maiergraben compared to 2752 Ind./\SI{}{\square\meter} at the Ois. The number of ephemeropterans is almost three times as high at the Ois as at Maiergraben too. In contrast, the number of occurring trichopterans is similar at both sites.



\begin{figure}[!htb]                              %% EPT Taxa
  \center
  \includegraphics[width=.75\linewidth]{images/EPT}                %% Width, Image file COLOR
  \caption{Number of EPT taxa collected at sampling sites.}               %% Figure Caption
  \label{fig:EPT}                                                       %% Figure label key
\end{figure}

The number of EPT-Taxa at the different choriotopes is pictured in \cref{fig:EPT_choriotope}. Except for macrolithal, where ephemeropterans have the highest number of taxa, trichopterans are the most diverse. The lowest number of different EPT-Taxa can be observed at the akal, whereas the mesolithal appears as the most diverse.

\begin{figure}[!htb]                              %% EPT choriotope
  \center
  \includegraphics[width=.75\linewidth]{images/EPT_choriotope}                %% Width, Image file COLOR
  \caption{Number of EPT taxon observed in different choriotopes.}               %% Figure Caption
  \label{fig:EPT_choriotope}                                                       %% Figure label key
\end{figure}




\subsubsection{Sensitive Taxa}\label{sec:sensitive_taxa_results}  %% Sensitive taxa
%\subsection{Sensitive Taxa}\label{sec:sensitive_taxa_results}  %% Sensitive taxa


At the Ois 27 sensitive taxa are present, which account for 56\% of the total taxa. In contrast only six sensitive taxa are present at the Maiergraben, which account for 25\% of the taxa as seen in the EcoProf results~(\cref{fig:ecoprof}). The percentages of sensitive individuals for both rivers are shown in~\cref{fig:sensitive_percentage}. Even though 56\% of all taxa of the Ois are classified as sensitive, they account for only 37\% of all individuals. At the Maiergraben only 4\% of all individuals are belonging to a sensitive group. Both sites cases indicate that sensitive individuals occur in lower numbers when compared to non-sensitive ones.


\begin{figure}[!htb]                            %% Percentage of sensitive taxa
  \center
  \includegraphics[width=.75\linewidth]{images/sensitive_percentage}                 %% Width, Image file COLOR
  \caption{Percentage of abundance of sensitive taxa at sample sites.}                      %% Figure Caption
  \label{fig:sensitive_percentage}                                                        %% Figure label key
\end{figure}


The number of occurring taxa among different mineral habitats is shown in \cref{fig:sensitive_choriotope}. The highest number of taxa was observed at mesolithal, whereas the lowest number of taxa was observed on akal (compare to \cref{fig:EPT_choriotope}). At the mineral habitats akal, meso-, and macrolithal more sensitive than non-sensitive taxa occur. On megalithal their number is equal and on technomegalithal more non-sensitive taxa are occurring.

%\begin{figure}[!htb]                            %% Sensitive taxa on choriotopes
%  \center
%  \includegraphics[width=.75\linewidth]{images/sensitive_choriotope}                 %% Width, Image file COLOR
%  \caption{Number of sensitive taxa found in the different substrate sizes.}                      %% Figure Caption
%  \label{fig:sensitive_choriotope}                                                        %% Figure label key
%\end{figure}


Moreover, \cref{fig:sensitive_individuals} shows the distribution of individuals among the different mineral habitats. Even though akal, meso-, and macrolithal were colonised by a higher number of sensitive taxa, more individuals of non-sensitive taxa occur, which gives the same picture as observed when comparing both rivers in general (\cref{fig:sensitive_percentage}). While more sensitive than non-sensitive individuals occur at the megalithal, the technomegalithal of the Maiergraben is colonised by many more individuals belonging to non-sensitive taxa.

%\begin{figure}[!htb]                            %% Sensitive percentage abundance on choriotopes
%  \center
%  \includegraphics[width=.75\linewidth]{images/sensitive_individuals}              %% Width, Image file COLOR
%  \caption{Number of sensitive and non sensitive individuals found in different substrate sizes.} %% Figure Caption
%  \label{fig:sensitive_individuals}                                                      %% Figure label key
%\end{figure}




\begin{figure}[!htb]                                                        %% Impacted site photo
\centering                                                                  %% Center the figures
\subcaptionbox{Number of sensitive taxa.\label{fig:sensitive_choriotope}}{   %% First subcaption
  \includegraphics[width=0.48\columnwidth]{images/sensitive_choriotope2}}  %% Set width and select first image COLOR
  \hfill                                                                                    %% Fill Blank space
\subcaptionbox{Number of sensitive and non sensitive individuals.\label{fig:sensitive_individuals}}{  %% Second subcaption
  \includegraphics[width=0.48\columnwidth]{images/sensitive_individuals2}}        %% Set width and second image COLOR
  \hspace*{\fill}                                                                           %% Fill blank space
\caption{Sensitive and non-sensitive taxa observed on different substrate sizes.}\label{fig:site_impacted}
    %% Capition for both figures
\end{figure}






























